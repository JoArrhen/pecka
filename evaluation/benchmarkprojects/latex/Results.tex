\documentclass[acmsmall, screen, review, nonacm]{acmart}

\usepackage{listings}
\usepackage{tikz}
\usepackage{multirow}
\usepackage{array}
\usetikzlibrary{arrows,decorations.pathmorphing,backgrounds,positioning,fit,shapes,matrix}
\newcommand{\JastAdd}{\textsc{Jast\-Add}}
\newcommand{\ExtendJ}{\textsc{ExtendJ}}
\newcommand{\IntraJ}{\textsc{IntraJ}}
\newcommand{\CAT}{\textsc{CAT}}
\newcommand{\eval}[2]{\textsc{#1}$_{\color{gray!70}{\pm#2}}$}
\newcommand{\basicStacked}{\textsc{Basic\-Stacked}}
\newcommand{\relaxedMonolithic}{\textsc{Relaxed\-Monolithic}}
\newcommand{\relaxedstacked}{\textsc{Relaxed\-Stacked}}
\newcommand{\relmon}{\textsc{rm}} %Abbreviation for relaxed Monolithic
\newcommand{\relstk}{\textsc{rs}} %Abbreviation for relaxed stacked
\newcommand{\code}[1]{\texttt{#1}}
\newcommand{\slowdown}[1]{$-$ #1 \color{red}{$\downarrow$}}
\newcommand{\speedup}[1]{$+$ #1 \color{teal}{$\uparrow$}}
\newcommand{\slowdownnew}[1]{$\times$ #1 \color{red}{$\downarrow$}}
\newcommand{\speedupnew}[1]{$\times$ #1 \color{teal}{$\uparrow$}}
\newcommand{\same}{\color{gray}{$\approx$}}

\newcommand{\extendjbaseline}{\ExtendJ$_{\relmon}$}
\newcommand{\extendjrelaxed}{\ExtendJ$_{\relstk}$}



\newcommand{\intrajbaseline}{\IntraJ$_{\relmon}$}
\newcommand{\intrajrelaxed}{\IntraJ$_{\relstk}$}


\newcommand{\percentwrt}[1]{\footnotesize
$\Delta$\%$^{\scriptscriptstyle\text{w.r.t}}_{\scriptscriptstyle\text{#1}}$%
}











\lstset{basicstyle=\scriptsize\ttfamily,
        escapeinside={/+}{+/},
        keywordstyle=[1]\bfseries}



\AtBeginDocument{%
  \providecommand\BibTeX{{%
    Bib\TeX}}}

%% Rights management information.  This information is sent to you
%% when you complete the rights form.  These commands have SAMPLE
%% values in them; it is your responsibility as an author to replace
%% the commands and values with those provided to you when you
%% complete the rights form.
\setcopyright{acmcopyright}
\copyrightyear{2018}
\acmYear{2018}
\acmDOI{XXXXXXX.XXXXXXX}

%% These commands are for a PROCEEDINGS abstract or paper.
\acmConference[PLDI]{}{June 24-28,
  2024}{Copenhagen, DK}


\acmPrice{15.00}
\acmISBN{978-1-4503-XXXX-X/18/06}




\begin{document}



\title{Artifact Evaluation: Efficient Demand Evaluation of Circular Attributes Using Static Analysis}


\author{Idriss Riouak}
\email{idriss.riouak@cs.lth.se}
\orcid{0000-0003-3520-2262}
\author{Niklas Fors}
\email{niklas.fors@cs.lth.se}
\orcid{0000-0003-2714-9457}
\author{Jesper {\"O}qvist}
\email{TBA}
\orcid{TBA}
\author{G{\"o}rel Hedin}
\email{gorel.hedin@cs.lth.se}
\orcid{0000-0002-3003-2623}
\affiliation{%
  \institution{Lund University}
  \streetaddress{Ole R\"{o}mers v\"{a}g 3}
  \city{Lund}
  \country{Sweden}
  \postcode{22363}
}



%%
%% By default, the full list of authors will be used in the page
%% headers. Often, this list is too long, and will overlap
%% other information printed in the page headers. This command allows
%% the author to define a more concise list
%% of authors' names for this purpose.
\renewcommand{\shortauthors}{Riouak et al.}


\maketitle

\section{Introduction}
This document presents the results of our artifact evaluation. Each section contains data generated by running our artifact and the corresponding results reported in the paper. The data is organized into tables and figures, providing insights into different aspects of our study.

\section{Dead Assignment Analysis (DAA)}
This section presents the results of the Dead Assignment Analysis.
\begin{table}[H]
	\begin{tabular}{|l|ccc|ccc|}
	\hline
	\multirow{3}{*}{\textsc{Benchmark}} & \multicolumn{3}{c|}{\textsc{Start up}} & \multicolumn{3}{c|}{\textsc{Steady State}} \\ \cline{2-7}
	& \multicolumn{1}{c|}{\intrajbaseline} & \multicolumn{2}{c|}{\intrajrelaxed} & \multicolumn{1}{c|}{\intrajbaseline} & \multicolumn{2}{c|}{\intrajrelaxed} \\ \cline{2-7}
	& \multicolumn{1}{c|}{Time (s)} & \multicolumn{1}{c|}{Time (s)} & \textsc{Speedup} & \multicolumn{1}{c|}{Time (s)} & \multicolumn{1}{c|}{Time (s)} & \textsc{Speedup} \\ \hline
	\end{tabular}
    \caption{\label{tab:daa} (NEW) Evaluation results for \emph{Dead Assignment Analysis} computing performance, comparing the start-up and steady-state execution times of the \intrajbaseline{} and \intrajrelaxed{}. Times are reported in seconds and speedup is reported as a ratio relative to the \intrajbaseline{} for both start-up and steady-state phases. }
    
\end{table}
\begin{table}[H]
	\begin{tabular}{|l|ccc|ccc|}
	\hline
	\multirow{3}{*}{\textsc{Benchmark}} & \multicolumn{3}{c|}{\textsc{Start up}} & \multicolumn{3}{c|}{\textsc{Steady State}} \\ \cline{2-7}
	& \multicolumn{1}{c|}{\intrajbaseline} & \multicolumn{2}{c|}{\intrajrelaxed} & \multicolumn{1}{c|}{\intrajbaseline} & \multicolumn{2}{c|}{\intrajrelaxed} \\ \cline{2-7}
	& \multicolumn{1}{c|}{Time (s)} & \multicolumn{1}{c|}{Time (s)} & \textsc{Speedup} & \multicolumn{1}{c|}{Time (s)} & \multicolumn{1}{c|}{Time (s)} & \textsc{Speedup} \\ \hline
	\code{commons-cli} & \multicolumn{1}{c|}{\eval{0.64}{nan}} & \multicolumn{1}{c|}{\eval{0.57}{nan}} & \speedupnew{1.11} & \multicolumn{1}{c|}{\eval{0.17}{nan}} & \multicolumn{1}{c|}{\eval{0.14}{nan}} & \speedupnew{1.26} \\ \hline
	\code{jackson-dataformat-xml} & \multicolumn{1}{c|}{\eval{2.00}{nan}} & \multicolumn{1}{c|}{\eval{1.97}{nan}} & \same{} & \multicolumn{1}{c|}{\eval{0.88}{nan}} & \multicolumn{1}{c|}{\eval{0.73}{nan}} & \speedupnew{1.22} \\ \hline
	\code{commons-jxpath} & \multicolumn{1}{c|}{\eval{1.35}{nan}} & \multicolumn{1}{c|}{\eval{1.33}{nan}} & \same{} & \multicolumn{1}{c|}{\eval{0.50}{nan}} & \multicolumn{1}{c|}{\eval{0.47}{nan}} & \speedupnew{1.06} \\ \hline
	\code{antlr-2.7.2} & \multicolumn{1}{c|}{\eval{1.94}{nan}} & \multicolumn{1}{c|}{\eval{1.81}{nan}} & \speedupnew{1.07} & \multicolumn{1}{c|}{\eval{0.77}{nan}} & \multicolumn{1}{c|}{\eval{0.76}{nan}} & \same{} \\ \hline
	\code{jackson-core} & \multicolumn{1}{c|}{\eval{2.57}{nan}} & \multicolumn{1}{c|}{\eval{3.01}{nan}} & \slowdownnew{0.85} & \multicolumn{1}{c|}{\eval{1.39}{nan}} & \multicolumn{1}{c|}{\eval{1.61}{nan}} & \slowdownnew{0.86} \\ \hline
	\code{pmd-4.2.5} & \multicolumn{1}{c|}{\eval{3.26}{nan}} & \multicolumn{1}{c|}{\eval{3.33}{nan}} & \slowdownnew{0.98} & \multicolumn{1}{c|}{\eval{1.55}{nan}} & \multicolumn{1}{c|}{\eval{1.59}{nan}} & \slowdownnew{0.97} \\ \hline
	\code{joda-time} & \multicolumn{1}{c|}{\eval{4.95}{nan}} & \multicolumn{1}{c|}{\eval{5.79}{nan}} & \slowdownnew{0.85} & \multicolumn{1}{c|}{\eval{2.94}{nan}} & \multicolumn{1}{c|}{\eval{4.22}{nan}} & \slowdownnew{0.70} \\ \hline
	\code{jfreechart-1.0.0} & \multicolumn{1}{c|}{\eval{4.10}{nan}} & \multicolumn{1}{c|}{\eval{6.61}{nan}} & \slowdownnew{0.62} & \multicolumn{1}{c|}{\eval{2.35}{nan}} & \multicolumn{1}{c|}{\eval{4.40}{nan}} & \slowdownnew{0.54} \\ \hline
	\code{fop-0.95} & \multicolumn{1}{c|}{\eval{4.41}{nan}} & \multicolumn{1}{c|}{\eval{4.67}{nan}} & \slowdownnew{0.94} & \multicolumn{1}{c|}{\eval{2.26}{nan}} & \multicolumn{1}{c|}{\eval{2.69}{nan}} & \slowdownnew{0.84} \\ \hline
	\end{tabular}
    \caption{\label{tab:daa} (NEW) Evaluation results for \emph{Dead Assignment Analysis} computing performance, comparing the start-up and steady-state execution times of the \intrajbaseline{} and \intrajrelaxed{}. Times are reported in seconds and speedup is reported as a ratio relative to the \intrajbaseline{} for both start-up and steady-state phases. }
    
\end{table}



\section{Null-pointer Exception Analysis (NPA)}
This section explores the results of the Null-pointer Exception Analysis.
\begin{table}[H]
	\begin{tabular}{|l|ccc|ccc|}
	\hline
	\multirow{3}{*}{\textsc{Benchmark}} & \multicolumn{3}{c|}{\textsc{Start up}} & \multicolumn{3}{c|}{\textsc{Steady State}} \\ \cline{2-7}
	& \multicolumn{1}{c|}{\intrajbaseline} & \multicolumn{2}{c|}{\intrajrelaxed} & \multicolumn{1}{c|}{\intrajbaseline} & \multicolumn{2}{c|}{\intrajrelaxed} \\ \cline{2-7}
	& \multicolumn{1}{c|}{Time (s)} & \multicolumn{1}{c|}{Time (s)} & \textsc{Speedup} & \multicolumn{1}{c|}{Time (s)} & \multicolumn{1}{c|}{Time (s)} & \textsc{Speedup} \\ \hline
	\end{tabular}
    \caption{\label{tab:npa} (NEW) Evaluation results for \emph{Null-Pointer Analysis} computing performance, comparing the start-up and steady-state execution times of the \intrajbaseline{} and \intrajrelaxed{}. Times are reported in seconds and speedup is reported as a ratio relative to the \intrajbaseline{} for both start-up and steady-state phases. Results are considered equivalent ({\same{}}) if the ratio is within the range [0.98, 1.02].}
    
\end{table}
\begin{table}[H]
	\begin{tabular}{|l|ccc|ccc|}
	\hline
	\multirow{3}{*}{\textsc{Benchmark}} & \multicolumn{3}{c|}{\textsc{Start up}} & \multicolumn{3}{c|}{\textsc{Steady State}} \\ \cline{2-7}
	& \multicolumn{1}{c|}{\intrajbaseline} & \multicolumn{2}{c|}{\intrajrelaxed} & \multicolumn{1}{c|}{\intrajbaseline} & \multicolumn{2}{c|}{\intrajrelaxed} \\ \cline{2-7}
	& \multicolumn{1}{c|}{Time (s)} & \multicolumn{1}{c|}{Time (s)} & \textsc{Speedup} & \multicolumn{1}{c|}{Time (s)} & \multicolumn{1}{c|}{Time (s)} & \textsc{Speedup} \\ \hline
	\code{commons-cli} & \multicolumn{1}{c|}{\eval{1.11}{nan}} & \multicolumn{1}{c|}{\eval{1.04}{nan}} & \speedupnew{1.07} & \multicolumn{1}{c|}{\eval{0.45}{nan}} & \multicolumn{1}{c|}{\eval{0.43}{nan}} & \speedupnew{1.06} \\ \hline
	\code{jackson-dataformat-xml} & \multicolumn{1}{c|}{\eval{2.51}{nan}} & \multicolumn{1}{c|}{\eval{2.38}{nan}} & \speedupnew{1.05} & \multicolumn{1}{c|}{\eval{1.18}{nan}} & \multicolumn{1}{c|}{\eval{1.25}{nan}} & \slowdownnew{0.94} \\ \hline
	\code{commons-jxpath} & \multicolumn{1}{c|}{\eval{2.03}{nan}} & \multicolumn{1}{c|}{\eval{1.73}{nan}} & \speedupnew{1.17} & \multicolumn{1}{c|}{\eval{0.78}{nan}} & \multicolumn{1}{c|}{\eval{0.66}{nan}} & \speedupnew{1.17} \\ \hline
	\code{antlr-2.7.2} & \multicolumn{1}{c|}{\eval{2.47}{nan}} & \multicolumn{1}{c|}{\eval{2.29}{nan}} & \speedupnew{1.08} & \multicolumn{1}{c|}{\eval{1.16}{nan}} & \multicolumn{1}{c|}{\eval{1.13}{nan}} & \speedupnew{1.03} \\ \hline
	\code{jackson-core} & \multicolumn{1}{c|}{\eval{3.81}{nan}} & \multicolumn{1}{c|}{\eval{3.79}{nan}} & \same{} & \multicolumn{1}{c|}{\eval{2.34}{nan}} & \multicolumn{1}{c|}{\eval{2.38}{nan}} & \same{} \\ \hline
	\code{pmd-4.2.5} & \multicolumn{1}{c|}{\eval{4.63}{nan}} & \multicolumn{1}{c|}{\eval{4.35}{nan}} & \speedupnew{1.06} & \multicolumn{1}{c|}{\eval{2.42}{nan}} & \multicolumn{1}{c|}{\eval{2.43}{nan}} & \same{} \\ \hline
	\code{joda-time} & \multicolumn{1}{c|}{\eval{7.38}{nan}} & \multicolumn{1}{c|}{\eval{7.66}{nan}} & \slowdownnew{0.96} & \multicolumn{1}{c|}{\eval{4.89}{nan}} & \multicolumn{1}{c|}{\eval{5.10}{nan}} & \slowdownnew{0.96} \\ \hline
	\code{jfreechart-1.0.0} & \multicolumn{1}{c|}{\eval{8.36}{nan}} & \multicolumn{1}{c|}{\eval{8.21}{nan}} & \same{} & \multicolumn{1}{c|}{\eval{5.95}{nan}} & \multicolumn{1}{c|}{\eval{6.13}{nan}} & \slowdownnew{0.97} \\ \hline
	\code{fop-0.95} & \multicolumn{1}{c|}{\eval{6.04}{nan}} & \multicolumn{1}{c|}{\eval{6.20}{nan}} & \slowdownnew{0.98} & \multicolumn{1}{c|}{\eval{3.86}{nan}} & \multicolumn{1}{c|}{\eval{3.73}{nan}} & \speedupnew{1.03} \\ \hline
	\end{tabular}
    \caption{\label{tab:npa} (NEW) Evaluation results for \emph{Null-Pointer Analysis} computing performance, comparing the start-up and steady-state execution times of the \intrajbaseline{} and \intrajrelaxed{}. Times are reported in seconds and speedup is reported as a ratio relative to the \intrajbaseline{} for both start-up and steady-state phases. Results are considered equivalent ({\same{}}) if the ratio is within the range [0.98, 1.02].}
    
\end{table}



\section{On-Demand Analysis}

This section presents a comparative analysis of IntraJ's steady-state execution time. The figures illustrate the average execution time for both Dead Assignment Analysis and Null-pointer Exception Analysis. The study is conducted on 10, 20, 50, 100, and 200 methods randomly selected from each benchmark.
\begin{figure}[H]
  \centering
  % \includegraphics[width=0.9\textwidth]{ondemand_analysis_time_vs_methods.png} 
  \caption{Comparative Analysis of IntraJ's Steady-State Execution Time: A Study on 10, 20, 50, 100, and 200 Methods Randomly Selected from Each Benchmark. 
  The left figure shows the average execution time for \emph{Dead Assignment Analysis}. The right figure shows the average execution time for \emph{Null-pointer Exception analysis}.}
\end{figure}


\begin{figure}[H]
  \centering
  % \includegraphics[width=0.9\textwidth]{ondemand_analysis_time_vs_methods.png} 
  \caption{(Paper reference) Comparative Analysis of IntraJ's Steady-State Execution Time: A Study on 10, 20, 50, 100, and 200 Methods Randomly Selected from Each Benchmark. 
  The left figure shows the average execution time for \emph{Dead Assignment Analysis}. The right figure shows the average execution time for \emph{Null-pointer Exception analysis}.}
\end{figure}

\section{ExtendJ Analysis}
This section discusses the results of the ExtendJ Analysis.
\begin{table}[H]
	\begin{tabular}{|l|ccc|ccc|}
	\hline
	\multirow{3}{*}{\textsc{Benchmark}} & \multicolumn{3}{c|}{\textsc{Start up}} & \multicolumn{3}{c|}{\textsc{Steady State}} \\ \cline{2-7}
	& \multicolumn{1}{c|}{\intrajbaseline} & \multicolumn{2}{c|}{\intrajrelaxed} & \multicolumn{1}{c|}{\intrajbaseline} & \multicolumn{2}{c|}{\intrajrelaxed} \\ \cline{2-7}
	& \multicolumn{1}{c|}{Time (s)} & \multicolumn{1}{c|}{Time (s)} & \textsc{Speedup} & \multicolumn{1}{c|}{Time (s)} & \multicolumn{1}{c|}{Time (s)} & \textsc{Speedup} \\ \hline
	\code{commons-cli} & \multicolumn{1}{c|}{\eval{0.95}{0.69}} & \multicolumn{1}{c|}{\eval{0.95}{0.25}} & \same{} & \multicolumn{1}{c|}{\eval{0.48}{0.56}} & \multicolumn{1}{c|}{\eval{0.42}{0.28}} & \speedupnew{1.14} \\ \hline
	\code{commons-jxpath} & \multicolumn{1}{c|}{\eval{1.61}{0.84}} & \multicolumn{1}{c|}{\eval{1.56}{0.26}} & \speedupnew{1.03} & \multicolumn{1}{c|}{\eval{0.79}{0.33}} & \multicolumn{1}{c|}{\eval{0.86}{0.39}} & \slowdownnew{0.91} \\ \hline
	\code{jackson-core} & \multicolumn{1}{c|}{\eval{3.22}{2.46}} & \multicolumn{1}{c|}{\eval{3.08}{2.02}} & \speedupnew{1.05} & \multicolumn{1}{c|}{\eval{1.62}{1.04}} & \multicolumn{1}{c|}{\eval{1.64}{0.86}} & \same{} \\ \hline
	\end{tabular}
    \caption{\label{tab:extendj} Performance comparison of ExtendJ’s execution time during startup and steady state.}
    
\end{table}
\begin{table}[H]
	\begin{tabular}{|l|ccc|ccc|}
	\hline
	\multirow{3}{*}{\textsc{Benchmark}} & \multicolumn{3}{c|}{\textsc{Start up}} & \multicolumn{3}{c|}{\textsc{Steady State}} \\ \cline{2-7}
	& \multicolumn{1}{c|}{\intrajbaseline} & \multicolumn{2}{c|}{\intrajrelaxed} & \multicolumn{1}{c|}{\intrajbaseline} & \multicolumn{2}{c|}{\intrajrelaxed} \\ \cline{2-7}
	& \multicolumn{1}{c|}{Time (s)} & \multicolumn{1}{c|}{Time (s)} & \textsc{Speedup} & \multicolumn{1}{c|}{Time (s)} & \multicolumn{1}{c|}{Time (s)} & \textsc{Speedup} \\ \hline
	\code{commons-cli} & \multicolumn{1}{c|}{\eval{0.93}{nan}} & \multicolumn{1}{c|}{\eval{0.94}{nan}} & \same{} & \multicolumn{1}{c|}{\eval{0.87}{nan}} & \multicolumn{1}{c|}{\eval{0.90}{nan}} & \slowdownnew{0.96} \\ \hline
	\code{jackson-dataformat-xml} & \multicolumn{1}{c|}{\eval{2.15}{nan}} & \multicolumn{1}{c|}{\eval{2.07}{nan}} & \speedupnew{1.04} & \multicolumn{1}{c|}{\eval{2.05}{nan}} & \multicolumn{1}{c|}{\eval{2.09}{nan}} & \same{} \\ \hline
	\code{commons-jxpath} & \multicolumn{1}{c|}{\eval{1.63}{nan}} & \multicolumn{1}{c|}{\eval{1.53}{nan}} & \speedupnew{1.06} & \multicolumn{1}{c|}{\eval{1.57}{nan}} & \multicolumn{1}{c|}{\eval{1.63}{nan}} & \slowdownnew{0.96} \\ \hline
	\code{antlr-2.7.2} & \multicolumn{1}{c|}{\eval{1.82}{nan}} & \multicolumn{1}{c|}{\eval{1.90}{nan}} & \slowdownnew{0.96} & \multicolumn{1}{c|}{\eval{1.90}{nan}} & \multicolumn{1}{c|}{\eval{1.76}{nan}} & \speedupnew{1.08} \\ \hline
	\code{jackson-core} & \multicolumn{1}{c|}{\eval{3.24}{nan}} & \multicolumn{1}{c|}{\eval{3.05}{nan}} & \speedupnew{1.06} & \multicolumn{1}{c|}{\eval{3.03}{nan}} & \multicolumn{1}{c|}{\eval{2.99}{nan}} & \same{} \\ \hline
	\code{pmd-4.2.5} & \multicolumn{1}{c|}{\eval{4.06}{nan}} & \multicolumn{1}{c|}{\eval{3.89}{nan}} & \speedupnew{1.04} & \multicolumn{1}{c|}{\eval{3.88}{nan}} & \multicolumn{1}{c|}{\eval{3.79}{nan}} & \speedupnew{1.02} \\ \hline
	\code{joda-time} & \multicolumn{1}{c|}{\eval{4.77}{nan}} & \multicolumn{1}{c|}{\eval{4.68}{nan}} & \same{} & \multicolumn{1}{c|}{\eval{4.67}{nan}} & \multicolumn{1}{c|}{\eval{4.57}{nan}} & \speedupnew{1.02} \\ \hline
	\code{jfreechart-1.0.0} & \multicolumn{1}{c|}{\eval{4.07}{nan}} & \multicolumn{1}{c|}{\eval{4.15}{nan}} & \same{} & \multicolumn{1}{c|}{\eval{4.15}{nan}} & \multicolumn{1}{c|}{\eval{4.12}{nan}} & \same{} \\ \hline
	\code{fop-0.95} & \multicolumn{1}{c|}{\eval{5.51}{nan}} & \multicolumn{1}{c|}{\eval{5.17}{nan}} & \speedupnew{1.06} & \multicolumn{1}{c|}{\eval{4.97}{nan}} & \multicolumn{1}{c|}{\eval{5.13}{nan}} & \slowdownnew{0.97} \\ \hline
	\end{tabular}
    \caption{\label{tab:extendj} Performance comparison of ExtendJ’s execution time during startup and steady state.}
    
\end{table}



\end{document}
\endinput
